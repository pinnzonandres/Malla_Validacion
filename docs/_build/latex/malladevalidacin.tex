%% Generated by Sphinx.
\def\sphinxdocclass{report}
\documentclass[letterpaper,10pt,spanish]{sphinxmanual}
\ifdefined\pdfpxdimen
   \let\sphinxpxdimen\pdfpxdimen\else\newdimen\sphinxpxdimen
\fi \sphinxpxdimen=.75bp\relax
\ifdefined\pdfimageresolution
    \pdfimageresolution= \numexpr \dimexpr1in\relax/\sphinxpxdimen\relax
\fi
%% let collapsible pdf bookmarks panel have high depth per default
\PassOptionsToPackage{bookmarksdepth=5}{hyperref}

\PassOptionsToPackage{booktabs}{sphinx}
\PassOptionsToPackage{colorrows}{sphinx}

\PassOptionsToPackage{warn}{textcomp}
\usepackage[utf8]{inputenc}
\ifdefined\DeclareUnicodeCharacter
% support both utf8 and utf8x syntaxes
  \ifdefined\DeclareUnicodeCharacterAsOptional
    \def\sphinxDUC#1{\DeclareUnicodeCharacter{"#1}}
  \else
    \let\sphinxDUC\DeclareUnicodeCharacter
  \fi
  \sphinxDUC{00A0}{\nobreakspace}
  \sphinxDUC{2500}{\sphinxunichar{2500}}
  \sphinxDUC{2502}{\sphinxunichar{2502}}
  \sphinxDUC{2514}{\sphinxunichar{2514}}
  \sphinxDUC{251C}{\sphinxunichar{251C}}
  \sphinxDUC{2572}{\textbackslash}
\fi
\usepackage{cmap}
\usepackage[T1]{fontenc}
\usepackage{amsmath,amssymb,amstext}
\usepackage{babel}



\usepackage{tgtermes}
\usepackage{tgheros}
\renewcommand{\ttdefault}{txtt}



\usepackage[Sonny]{fncychap}
\ChNameVar{\Large\normalfont\sffamily}
\ChTitleVar{\Large\normalfont\sffamily}
\usepackage{sphinx}

\fvset{fontsize=auto}
\usepackage{geometry}


% Include hyperref last.
\usepackage{hyperref}
% Fix anchor placement for figures with captions.
\usepackage{hypcap}% it must be loaded after hyperref.
% Set up styles of URL: it should be placed after hyperref.
\urlstyle{same}


\usepackage{sphinxmessages}




\title{Malla de Validación}
\date{08 de noviembre de 2023}
\release{0.1}
\author{Wilson Andrés Pinzón}
\newcommand{\sphinxlogo}{\vbox{}}
\renewcommand{\releasename}{Versión}
\makeindex
\begin{document}

\ifdefined\shorthandoff
  \ifnum\catcode`\=\string=\active\shorthandoff{=}\fi
  \ifnum\catcode`\"=\active\shorthandoff{"}\fi
\fi

\pagestyle{empty}
\sphinxmaketitle
\pagestyle{plain}
\sphinxtableofcontents
\pagestyle{normal}
\phantomsection\label{\detokenize{index::doc}}


\sphinxAtStartPar
Proyecto de Python con las clases y modulos requeridos para la validación de los datos adquiridos a través de los formularios de RIT


\chapter{Clase Validación}
\label{\detokenize{index:clase-validacion}}
\sphinxAtStartPar
La clase validación se refiere al objeto que va almacenar la información relacionada a la validación de la información:
\begin{itemize}
\item {} 
\sphinxAtStartPar
Conjunto de Datos

\item {} 
\sphinxAtStartPar
Malla de Validación (JSON)

\item {} 
\sphinxAtStartPar
Resultados de la validación

\end{itemize}
\index{validacion (clase en MallaValidacionRuta)@\spxentry{validacion}\spxextra{clase en MallaValidacionRuta}}

\begin{fulllineitems}
\phantomsection\label{\detokenize{index:MallaValidacionRuta.validacion}}
\pysigstartsignatures
\pysiglinewithargsret{\sphinxbfcode{\sphinxupquote{class\DUrole{w}{ }}}\sphinxcode{\sphinxupquote{MallaValidacionRuta.}}\sphinxbfcode{\sphinxupquote{validacion}}}{\sphinxparam{\DUrole{n}{nombre\_api}\DUrole{p}{:}\DUrole{w}{ }\DUrole{n}{str}}\sphinxparamcomma \sphinxparam{\DUrole{n}{json\_malla}\DUrole{p}{:}\DUrole{w}{ }\DUrole{n}{bool}}\sphinxparamcomma \sphinxparam{\DUrole{n}{nombre\_malla}\DUrole{p}{:}\DUrole{w}{ }\DUrole{n}{str}}\sphinxparamcomma \sphinxparam{\DUrole{n}{ruta}\DUrole{p}{:}\DUrole{w}{ }\DUrole{n}{str}}}{}
\pysigstopsignatures
\sphinxAtStartPar
Esta clase tiene como objetivo descargar y validar los datos a partir de la malla de validación entregada
\begin{description}
\sphinxlineitem{Atributos:}\begin{itemize}
\item {} 
\sphinxAtStartPar
nombre\_api (str) : Nombre del archivo json config donde se encuentra la puerta de acceso a los datos por medio del API de RIT

\item {} 
\sphinxAtStartPar
json\_malla (bool) : Booleano que le indica a la clase si la malla de validación ya se encuentra en formato JSON o se debe crear a partir del archivo Excel

\item {} 
\sphinxAtStartPar
nombre\_malla (str) : Nombre que identifica a la malla de validación, ya sea de tipo JSON o de tipo Excel (Ambos archivos deben tener el mismo nombre)

\item {} 
\sphinxAtStartPar
ruta (str) : Cadena de texto que ubica la ruta a la carpeta del proyecto o al repositorio

\end{itemize}

\end{description}

\end{fulllineitems}



\chapter{Validación de la Información}
\label{\detokenize{index:validacion-de-la-informacion}}
\sphinxAtStartPar
Para realizar el proceso de validación del conjunto de datos, una vez se ha definido la clase validación con los parámetros definidos, se ejecuta el método validar\_datos, este se va a encargar de descargar, normalizar y validar la información.
\index{validar\_datos() (en el módulo MallaValidacionRuta.validacion)@\spxentry{validar\_datos()}\spxextra{en el módulo MallaValidacionRuta.validacion}}

\begin{fulllineitems}
\phantomsection\label{\detokenize{index:MallaValidacionRuta.validacion.validar_datos}}
\pysigstartsignatures
\pysiglinewithargsret{\sphinxcode{\sphinxupquote{MallaValidacionRuta.validacion.}}\sphinxbfcode{\sphinxupquote{validar\_datos}}}{\sphinxparam{\DUrole{n}{self}}}{}
\pysigstopsignatures
\sphinxAtStartPar
Método de la clase validación que se encarga de validar la información del conjunto de datos a través de la malla de validación obtenida
\begin{quote}\begin{description}
\sphinxlineitem{Devuelve}
\sphinxAtStartPar
Devuelve tres dataframes de Pandas con los resultados de la validación, la validación general,
los registros validos y los registros no validos

\sphinxlineitem{Tipo del valor devuelto}
\sphinxAtStartPar
(pd.DataFrame, pd.Dataframe, pd.Dataframe)

\end{description}\end{quote}

\end{fulllineitems}



\chapter{Funciones Adicionales}
\label{\detokenize{index:funciones-adicionales}}

\section{Crear Malla de Validación en formato JSON}
\label{\detokenize{index:crear-malla-de-validacion-en-formato-json}}
\sphinxAtStartPar
En el caso que la estructura de la malla de validación se encuentre en formato Excel, existe la función create\_json\_malla del modulo malla\_functions que le va a permitir crear el archivo json con la malla de validación.
\index{create\_json\_malla() (en el módulo malla\_functions)@\spxentry{create\_json\_malla()}\spxextra{en el módulo malla\_functions}}

\begin{fulllineitems}
\phantomsection\label{\detokenize{index:malla_functions.create_json_malla}}
\pysigstartsignatures
\pysiglinewithargsret{\sphinxcode{\sphinxupquote{malla\_functions.}}\sphinxbfcode{\sphinxupquote{create\_json\_malla}}}{\sphinxparam{\DUrole{n}{name\_malla}\DUrole{p}{:}\DUrole{w}{ }\DUrole{n}{str}}\sphinxparamcomma \sphinxparam{\DUrole{n}{ruta}\DUrole{p}{:}\DUrole{w}{ }\DUrole{n}{str}}}{}
\pysigstopsignatures
\sphinxAtStartPar
Función que crea el archivo JSON para la estructura de la malla de validación a partir de un archivo Excel y lo exporta en la ruta establecida
\begin{quote}\begin{description}
\sphinxlineitem{Parámetros}\begin{itemize}
\item {} 
\sphinxAtStartPar
\sphinxstyleliteralstrong{\sphinxupquote{name\_malla}} (\sphinxstyleliteralemphasis{\sphinxupquote{\sphinxhyphen{}}}) \textendash{} Nombre del archivo excel que contiene la estructura de la malla

\item {} 
\sphinxAtStartPar
\sphinxstyleliteralstrong{\sphinxupquote{ruta}} (\sphinxstyleliteralemphasis{\sphinxupquote{\sphinxhyphen{}}}) \textendash{} Ruta de la dirección de la carpeta del proyecto

\end{itemize}

\sphinxlineitem{Devuelve}
\sphinxAtStartPar
Crea un archivo JSON en la carpeta \sphinxtitleref{data/validation\_json} dentro de la ruta específicada

\end{description}\end{quote}

\end{fulllineitems}



\section{Malla de Validación}
\label{\detokenize{index:id1}}
\sphinxAtStartPar
La siguiente función es la encargada de realizar la validación de los datos a partir de las condiciones de la Malla de Validación
\index{malla\_validacion() (en el módulo malla\_functions)@\spxentry{malla\_validacion()}\spxextra{en el módulo malla\_functions}}

\begin{fulllineitems}
\phantomsection\label{\detokenize{index:malla_functions.malla_validacion}}
\pysigstartsignatures
\pysiglinewithargsret{\sphinxcode{\sphinxupquote{malla\_functions.}}\sphinxbfcode{\sphinxupquote{malla\_validacion}}}{\sphinxparam{\DUrole{n}{data}\DUrole{p}{:}\DUrole{w}{ }\DUrole{n}{DataFrame}}\sphinxparamcomma \sphinxparam{\DUrole{n}{guia\_validacion}\DUrole{p}{:}\DUrole{w}{ }\DUrole{n}{dict}}}{}
\pysigstopsignatures
\sphinxAtStartPar
Realiza la validación de datos basada en la malla de validación.
\begin{quote}\begin{description}
\sphinxlineitem{Parámetros}\begin{itemize}
\item {} 
\sphinxAtStartPar
\sphinxstyleliteralstrong{\sphinxupquote{data}} (\sphinxstyleliteralemphasis{\sphinxupquote{\sphinxhyphen{}}}) \textendash{} DataFrame de datos a validar.

\item {} 
\sphinxAtStartPar
\sphinxstyleliteralstrong{\sphinxupquote{guia\_validacion}} (\sphinxstyleliteralemphasis{\sphinxupquote{\sphinxhyphen{}}}) \textendash{} Malla de validación que especifica las condiciones y valores para cada columna.

\end{itemize}

\sphinxlineitem{Devuelve}
\sphinxAtStartPar
Devuelve tres dataFrames de pandas con los resultados de la validación, incluyendo columnas de errores y puntuación de validación

\sphinxlineitem{Tipo del valor devuelto}
\sphinxAtStartPar
pd.DataFrame, pd.DataFrame, pd.DataFrame

\end{description}\end{quote}

\end{fulllineitems}




\renewcommand{\indexname}{Índice}
\printindex
\end{document}